\section{MAD: Erlang Build and Deploy Tool}

\subsection{History}

We realized no matter how perfect your libraries are,
the comfort and easy come mostly from developing tools.
Everything got started when \footahref{https://github.com/proger}{Vladimir~Kirillov} decided to
replace Rusty's sync beam reloader. As you know sync uses
filesystem polling which is neither energy-efficient nor elegant. Also
sync is only able to recompile separate modules while
common use-case in N2O is to recompile DTL templates
and LESS/SCSS stylesheets. That is why we need to recompile
the whole project. That's the story how active emerged.
Under the hood active is a client subscriber
of fs library, native filesystem listener for Linux, Windows and Mac.

De-facto standard in Erlang world is rebar.
We love rebar interface despite its implementation.
First we plugged rebar into active and then decided to drop its support.
Later switched to Makefile-based build tool otp.mk.
And of course it is slow, especially in cold recompilation.
It was designed to be a stand-alone tool, so it has some
glitches while using as embedded library.

The idea to build rebar replacement was shared in space for a long time.
The best minimal approach was picked up by \footahref{https://github.com/s1n4}{Sina~Samavati},
he implemented the first prototype called mad. Initially mad
was able to compile DTL templated, YECC files, escript (like
bundled in gproc), also it was support caching with side-effects.
After a month I forked mad and took over the development under the same name.

\vspace{1\baselineskip}
\begin{lstlisting}[caption=Example of building N2O sample]
                                   Cold       Hot
    rebar get-deps compile         53.156s    4.714s
    mad deps compile               54.097s    0.899s
\end{lstlisting}
\vspace{1\baselineskip}

\vspace{1\baselineskip}
\begin{lstlisting}[caption=Example of building Cowboy]
                                   Hot
    make (erlang.mk)               2.588s
    mad compile                    2.521s
\end{lstlisting}
\vspace{1\baselineskip}

\subsection{Introduction}

We were trying to make something minimalistic that fits out Web Stack.
Besides we wanted to use our knowledge of other build tools like lein, sbt etc.
Also for sure we tried sinan, ebt, Makefile-based scripts.

Synrc mad has a simple interface as follows:

\vspace{1\baselineskip}
\begin{lstlisting}
  BNF:
      invoke := mad params
      params := [] | run params
         run := command [ options ]
     command := app | lib | deps | compile | bundle
                run | stop | attach | repl
\end{lstlisting}
\vspace{1\baselineskip}

It seems to us more natural, you can specify random
commands set with different specifiers (options).

\subsection{Single-File Bundling}

The key feature of mad is ability to create single-file bundled web sites.
Thus making Joe's dream to boot like node.js come true.
This target escript is ready to run on Windows, Linux and Mac.

To make this possible we implemented a zip filesytem inside escript.
mad packages priv directories along with ebin and configs.
You can redefine each file in zip fs inside target
escript by creation the copy with same path locally near escript.
After launch all files are copied to ETS.
N2O also comes with custom cowboy static handler that is able to
read static files from this cached ETS filesystem.
Also bundle are compatible with active online realoading and recompilation.

\subsection{Templates}

mad also comes with N2O templates. So you can bootstrap a N2O-based site
just having a single copy of mad.

\vspace{1\baselineskip}
\begin{lstlisting}
    # mad app sample
    # cd sample
    # mad deps compile plan bundle web_app
\end{lstlisting}
\vspace{1\baselineskip}

After that you can just run ./web_app under Windows, Linux and
Mac and open \footahref{http://localhost:8000}{http://localhost:8000}.

\vspace{1\baselineskip}
\begin{lstlisting}
    C:\> escript web_app
    Applications: [kernel,stdlib,crypto,cowlib,ranch,cowboy,compiler,syntax_tools,
                   erlydtl,gproc,xmerl,n2o,n2o_sample,fs,active,mad,sh]
    Configuration: [{n2o,[{port,8000},{route,routes}]},
                    {kvs,[{dba,store_mnesia},
                          {schema,[kvs_user,kvs_acl,kvs_feed,kvs_subscription]}]}]
    Erlang/OTP 17 [erts-6.0] [64-bit] [smp:4:4] [async-threads:10] [kernel-poll:false]

    Eshell V6.0  (abort with ^G)
    1>
\end{lstlisting}
\vspace{1\baselineskip}

\subsection{Deploy}

mad is also supposed to be also a deploy tool with ability to
deploy not only to our resources like Erlang on Xen, Voxoz (LXC/Xen) but
also to Heroku and others.

\subsection{OTP Compliant}

mad supports rebar umbrella project structure.
Specifically two kinds of directory layouts:

\vspace{1\baselineskip}
\begin{lstlisting}[caption=Solution]
    ├── apps
    ├── deps
    ├── rebar.config
    └── sys.config
\end{lstlisting}
\vspace{1\baselineskip}

\vspace{1\baselineskip}
\begin{lstlisting}[caption=OTP Application]
    ├── deps
    ├── ebin
    ├── include
    ├── priv
    ├── src
    └── rebar.config
\end{lstlisting}
\vspace{1\baselineskip}

Also you can create OTP releases with reltool, rebar configure or relx.
Internally mad uses reltool to order your dependencies and generation of reltool.config.

\vspace{1\baselineskip}
\begin{lstlisting}
    # mad plan
    Ordered: [kernel,stdlib,mnesia,kvs,crypto,cowlib,ranch,
              cowboy,compiler,syntax_tools,erlydtl,gproc,
              xmerl,n2o,n2o_sample,fs,active,mad,rest,sh]
\end{lstlisting}
\vspace{1\baselineskip}
