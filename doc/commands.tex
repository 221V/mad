\section{Commands}

Synrc mad has a simple interface as follows:

\vspace{1\baselineskip}
\begin{lstlisting}
  BNF:
      invoke := mad params
      params := [] | run params
         run := command [ options ]
     command := app | lib | deps | compile | bundle
                start | stop | repl
\end{lstlisting}
\vspace{1\baselineskip}

It seems to us more natural, you can specify random
commands set with different specifiers (options).

\subsection{deps}

In rebar-like managers we are selecting deps from rebar.config:
\vspace{1\baselineskip}
\begin{lstlisting}
  {sub_dirs,["apps"]}.
  {deps_dir,"deps"}.
  {deps, [active,{nitro,"2.9"},{n2o,"2.9"}]}.
\end{lstlisting}
\vspace{1\baselineskip}


\subsection{compile, com}
Performs compilation of all known compilations backends in complilation profile of mad:
\vspace{1\baselineskip}
\begin{lstlisting}
    1. mad_app — app.src erlang templating.
    2. mad_dtl — DTL compiler.
    3. mad_erl — BEAM compiler.
    4. mad_port — for gcc cland and other native compilation.
    5. mad_script — .script file used in projects like gproc.
    6. mad_yecc/mad_leex — DSL language parser compilers.
    7. mad_upl — UPL compiler.
\end{lstlisting}
\vspace{1\baselineskip}

\subsection{pla}
Taking all dependecies and resolve boot sequence according to dependecy order. Storing this value in .applist.
\vspace{1\baselineskip}
\begin{lstlisting}
  Ordered: [kernel,stdlib,fs,ranch,crypto,compiler,syntax_tools,
            gproc,cowlib,cowboy,n2o,sample,active,erlydtl,jsone,
            mad,nitro,sh]
\end{lstlisting}
\vspace{1\baselineskip}

\subsection{repl}

\subsection{bundle}

\subsection{app}

